\documentclass{hagaki-doc}

\title{Hagaki クラス}
\author{HiroSpark}
\date{v0.3.0(2022/03/16)}

\begin{document}

\maketitle

\section{概要}

\code{Hagaki}クラスはハガキの宛名印刷を簡単に行うためのドキュメントクラスで、
Lua\TeX{}上で動作します。
ライセンスはMITです。

\begin{description}
  \item[Issues] \url{https://github.com/HiroSpark/hagaki/issues}
  \item[ソースコード] \url{https://github.com/HiroSpark/hagaki}
\end{description}

\section{インストール}

\href{https://github.com/HiroSpark/hagaki/releases/}{GitHub Releases}から
ソースコードをダウンロードし、\code{build.lua}のあるディレクトリで、

\begin{lstlisting}
$ l3build install
\end{lstlisting}

を実行してください。
\pkg{l3build}によって、ユーザーの\code{TEXMF}ツリーに
\code{hagaki.cls}がインストールされます。
アンインストールするには、同じディレクトリで、

\begin{lstlisting}
$ l3build uninstall
\end{lstlisting}

を実行します。
詳細は\pkg{l3build}のドキュメントを見てください。

\section{基本的な使い方}

\subsection{普通ハガキと年賀ハガキ}

\pkg{hagaki}クラスのデフォルトは、普通ハガキのレイアウトです。
年賀ハガキのレイアウトにするには、クラスオプションに\code{nenga}を指定してください。

\begin{lstlisting}
\documentclass[nenga]{hagaki}
\end{lstlisting}

\subsection{\cs{sender}と\cs{recipient}}

\cs{sender}は差出人の情報を、\cs{recipient}は宛先の情報を指定します。
\cs{recipient}は宛先の情報の指定とともに、宛名面の作成も行います。

どちらも\code{keyval}リストで郵便番号・住所・名前を指定します。
受け取る値の詳細は以下のようになっています。

\begin{tabularx}{\linewidth}{llX}
  \toprule
  \code{key} & 指定する値 & 注意 \\
  \midrule
  \code{postal\_code} & 郵便番号 & \code{-}が必要です。\\
  \code{name} & 名前 & 半角スペースで名字と名前を区切ってください。\\
  \code{address} & 住所 &
  組版の関係で、英数字は「\rotatebox[origin=c]{270}{1}」のようになってしまいます。
  それを避けたい場合は、数字を和文文字にするか、クラスオプションに\code{rensuji}を
  指定してください。\\
  \bottomrule
\end{tabularx}

例えば、以下のソースを処理すると、次のような宛名面が作成されます。

\begin{figure}
  \begin{minipage}{.6\linewidth}
    \lstinputlisting{example.tex}
  \end{minipage}
  \begin{minipage}{.4\linewidth}
    \image[width=\linewidth]{example.pdf}
  \end{minipage}
\end{figure}

\section{リファレンス}

\subsection{クラスオプション}

\begin{tabularx}{\linewidth}{lX}
  \toprule
  オプション & 説明 \\
  \midrule
  \code{nenga} & 年賀状用のレイアウトに変更します。\pkg{hagaki}のデフォルトレイアウトは通常ハガキです。\\
  \code{rensuji} & 住所の数字に対して、自動的に\cs{rensuji}を適用します。\pkg{lltjext}を読み込みます。\\
  \code{debug} & デバッグ用に、グリッドが表示されます。ページの右上を基準に、太い線が10mmごと、
    細い線が2mmごとに引かれます。\\
  \bottomrule
\end{tabularx}

\subsection{\cs{hagakisetup}}

\cs{hagakisetup\{<options>\}}でレイアウトをカスタマイズ出来ます。
\code{options}には\code{keyval}リストで以下の値を指定可能です。
この機能は実験的なため、将来的に変更される可能性があります。

\begin{tabularx}{\linewidth}{ll}
  \toprule
  key & 説明 \\
  \midrule
  \code{top\_line}                        & 宛先の住所と名前の上端 \\
  \code{bottom\_line}                     & 宛先の名前と差出人の住所と名前の下端 \\
  \code{recipient / address\_position}    & 宛先の住所の中心(y座標) \\
  \code{recipient / name\_position}       & 宛先の名前の中心(y座標) \\
  \code{recipient / name\_indent}         & 宛先の名前のインデント \\
  \code{recipient / family\_name\_width}  & 宛先の苗字に充てるスペース \\
  \code{sender / address\_position}       & 差出人の住所の中心(y座標) \\
  \code{sender / name\_position}          & 差出人の名前の中心(y座標) \\
  \bottomrule
\end{tabularx}

\subsection{\code{expl3}インターフェイス}

\code{hgk}モジュールとして提供されます。
この機能は実験的なため、将来的に変更される可能性があります。

\begin{tabularx}{\linewidth}{ll}
  \toprule
  関数 & 挿入箇所 \\
  \midrule
  \cs{hgk\_rcpt\_postal\_code\_style:}    & 宛先の郵便番号 \\
  \cs{hgk\_rcpt\_address\_style:}         & 宛先の住所 \\
  \cs{hgk\_rcpt\_name\_style:}            & 宛先の名前 \\
  \cs{hgk\_sender\_postal\_code\_style:}  & 差出人の郵便番号 \\
  \cs{hgk\_sender\_address\_style:}       & 差出人の住所 \\
  \cs{hgk\_sender\_name\_style:}          & 差出人の名前 \\
  \bottomrule
\end{tabularx}

\section{バージョン履歴}

\begin{description}
  \item[v0.3.0] 依存パッケージの削減と機能の改善を行いました。
    \leavevmode
    \begin{itemize}
      \item デフォルトフォントのロードを廃止
      \item \pkg{TikZ}及び\pkg{lua-visual-debug} への依存を廃止
      \item 宛名面の描画を\code{tikzpicture}環境から\code{picture}環境に変更
      \item \code{debug}オプション指定時に描画されるグリッドの改善
    \end{itemize}
  \item[v0.2.0] カスタマイズ関連を中心に、機能の追加・強化といくつかの破壊的変更を行いました。
    \leavevmode
    \begin{itemize}
      \item 年賀状用のレイアウトはクラスオプション\code{nenga}の指定が必要に
      \item \cs{hagakisetup}によるカスタマイズ機能の導入
      \item \pkg{expl3}インターフェイスによるコードの挿入
      \item 数字に\cs{rensuji}を自動的に適用するクラスオプション\code{rensuji}を追加
      \item デバッグ用のクラスオプション\code{debug}を追加
      \item 差出人・宛先が複数の場合に対応
    \end{itemize}
  \item[v0.1.0] 最初のリリース
\end{description}

\end{document}
